09:58 Introduction, Preamble

09:59 Background and AI Experience
Chair of Digital Pathology, no self prior experience with AI/ML, but image classifiers. Main field: image analysis

10:03 Showing Sample Output
you have to have original image - you can not see what is really positive here. Some small nuclei in image center are possibly tumor cells. 

10:06 Trust Scores
G: Negatively stained nuclei missed - should have proof of confidentiality of nuclei. Adapt output to threshold. Ki67 is bad example for AI output since it is not standardized. High inter-pathologist variability. T: What would it take to trust you? G: Good results. Confidentiality display is good.

10:15 Counterfactuals
G: Shows us the colour brown (size,shape, chromatine) for nuclei is important to Ki67. T: WHat are important factors for algo? G: Algo is looking for shape and darkness. I dont know which factors I am looking for, its experience. 2Axis CF is addon.

10:22 Prototypes
G: Shows me prototypes for pos neg examples. This and CF are valuable, Prototypes are most simple, reflects thinking of pathologist.

10:26 Concept Attribution
G: Prototype is more elaborate. Interpretation is most important - I dont know what intensity is meant here, both intensities are needed. If you have other info for differentiating between nuclei.
Need standardized vocabulary for staining intensity, where to apply. 
Daicam? standard for making mass annotations in images with key value pair (pos/neg,round/nround, ...) annotated in image. Shows what algo sees in image and how it expresses it. Not only problem of explanation, but also observation.
T: What are big challenges on this way? G: The biggest challenge is to start. Path need collaborative work, discuss what is understandable and what is not.

10:37 Saliency Maps
G: What is the combination of features in this image for making a decision. It could be used as additional tool for making me trust it. If I see that most salient objects are not satisfying me - I can tell whether it is looking at the right objects - top middle are out of focus. It is valuable in suite of tools, no improvements.

10:43 After
T: Risks of AI. G: Most important is trust, Path has choice of tools and stands for diagnosis. Pathologist has expectation and checks if AI solution is in line with that. AI should be included in training program for Path. T: What are Path expectations? G: Solve boring problems/routines - nobody counts the exact number of antibodies to proportion of 14.4 perc. Solve these problems in a more objective way. T: Different problems for trivial job and impossible problem to solve - with different requirements. T: How do you expect your colleague to show you? G: Show point in image and make annotations, explain what structure and so on has led me to my decision.
T: How would you train future paths? G: Its a theoretical discussion since few have digital path in their labs, and without digital slides there is no AI. In a few years we will introduce AI in residence training.

Outcome: 
-Don't trust prototypes - Tesla Autopilot traffic sign analogy, blue sky prototypes. Mom says we have lympocytes at home
-do we want to be observational or active with the questions? -> be observational first, in the end loop back and encourage introspection

Conclusion of most important points:

Prototypes
* Like counterfactuals, but with less information (see impact of order here)
* Gives me the same type of info, with less confidence. Do not know how I would interpret this without having seen counterfactuals
* Understand the ‘perfect’ +ve and -ve result
* Hard to imagine useful contexts, pathology is about diversity.
* CF: I can see the grades, nothing is black and white, prototypical example do not reflect the true result
* Useful in context: seeing prototypical results demonstrating key features (presence of nucleolus, cytoplasm), between very similar results > ie. prototypes of edge cases, boundary cases (tending toward CFs)
* Example: diff. Lymphocytes from plasmocytes, subtle difference, hard to distinguish at low power, even at high power non-trivial. Having prototypes helps lend confidence that the LC and PC (‘football’ pattern staining in latter) are being resolved from one another.

Profile
*Retired Pathologists
*Chair of Digital Pathology Comittee
*Engaged with AI since it is the core of Digital Pathology
*Connections to EMPAIA
*No deeper insights into ML technologies, but into broader topics

General
*Broad definition of AI according to EU, can also mean statistical classifiers
*Ki67 is not standardized and thus bad example for AI(?), since staining varies strongly - do we change staining or algorithm?
*Dont even have standardized vocabulary, first step for having standardised reports
*Daicon 222 standard for having key-value pair annotation
*expectations are to offload boring, routine tasks and objective, comparable results for these
*Few digital pathology present, its growing but still abstract
*Need to train Pathologists, also good for AI to do that

First Image
*have to see original image without the overlay

Trust Score
*need proof of confidentiality for positive and negative staining
*shows basis of decision and theoretically allows user to adapt output so that it matches Pathologists impression
*Should allow students to gauge reliability of result

Counterfactuals
*its good?
*2axis shows unclassifiable examples -> confirmation bias, falls in line with you own understanding
*2axis is nice addon, but not as simple

Prototypes
*easy understandable, most simple, helps trust
*shows that algo is concentrating on prototypical features -> builds trust
*more examples would be better -> Paths expect to be in the loop and dont give full autonomy

Concept Attribution
*more elaborated than Prototypes
*what intensity is meant here?
*more value would be added by different features for differentiating between tumor, nuclei and so on

Saliency Maps
*see wether decisions are made on right regions
*I see combination of features that were used to make a decision
