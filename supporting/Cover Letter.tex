Dear Editors of the special issue FGCS\_XAI4H
Please find enclosed our submission

"Trust and Trustworthiness: Assessing the usability of xAI in digital pathology"

Our submission is original, has neither been submitted nor is it under consideration to be submitted elsewhere and has exclusively been produced for your special issue. All authors declare that there is no conflict of interest, and that the paper does not raise any ethical issues. We think that this submission is a valuable work for the international scientific community and that your special issue is a perfect fit for the audience. I am happy to act as the corresponding author for all future communication on behalf of all co-authors.

With the recent scientific progress in the field of image based machine learning, new and promising ways for exploiting digitalized whole-slide-images have been demonstrated to drive the field into the new era of AI-powered precision diagnostics. Further specialization and a large amount of pathologists close to retirement drive the demand for solutions to increase productivity through automation. Translating proven AI-applications into medical practice requires a high level of reliability, control and trust, which the new, better performing but very complex models do not intrinsically provide. The use of explainable AI methods can help overcome this issue, but their application requires careful investigation of the context they are applied in, the methods being used and actual benefits they provide for the enduser.


We present a mixed-methods user study (survey & interviews) which explores the core factors driving the adoption of explainable AI in digital pathology. We investigated the effects of five major xAI methods and their application on a exemplary and specific use case of KI-67 tumor grading. This survey is to our knowledge the first of its kind to assess the interpretation and evaluation of xAI-generated explanations in the domain of AI-powered digital pathology by the actual target audience. We hope to provide valuable insights for ML researchers, AI developers and HCI designers to build xAI solutions in pathology, and hope that our research can serve as a template for other efforts in the area of image based diagnosis, healthcare and beyond. 

Core contributions are the detailed evaluation of the strengths and weakness of the most popular xAI classes, both qualitative and quantitative. We discuss overall strengths and pitfalls for xAI solutions in Digital Pathology, as well as requirements for their adoption.


